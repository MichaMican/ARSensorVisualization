\usepackage{listings}

\renewcommand{\lstlistingname}{Code}
\renewcommand{\lstlistlistingname}{Quellcodeausschnitte}

% Define TypeScript as language for listings
\lstdefinelanguage{TypeScript}{
	keywords={
		typeof, new, true, false,
		function, class, this,
		var, let, const, in, is,
		async, instanceof, private,
		public, readonly, constructor,
		declare, module
	},
	keywordstyle=\color{blue}\bfseries,
	otherkeywords={% Operators
		=>
	},
	ndkeywords={
		return, if, while, do, else, catch,
		switch, case, break, export,
		throw, implements, import, await,
		null, undefined, boolean, string,
		any, void, number
	},
	ndkeywordstyle=\color{violet}\bfseries,
	identifierstyle=\color{black},
	sensitive=false,
	comment=[l]{//},
	morecomment=[s]{/*}{*/},
	commentstyle=\color{ForestGreen},
	stringstyle=\color{red},
	morestring=[b]',
	morestring=[b]",
	morestring=[b]`
}

\lstdefinelanguage{CSharp}{
	keywords={
		abstract, event, new, struct, var,
		as, explicit, null, switch,
		base, extern, object, this,
		bool, false, operator, throw,
		break, finally, out, true,
		byte, fixed, override, try,
		case, float, params, typeof,
		catch, for, private, uint,
		char, foreach, protected, ulong,
		checked, goto, public, unchecked,
		class, if, readonly, unsafe,
		const, implicit, ref, ushort,
		continue, in, return, using,
		decimal, int, sbyte, virtual,
		default, interface, sealed, volatile,
		delegate, internal, short, void,
		do, is, sizeof, while,
		double, lock, stackalloc,
		else, long, static,
		enum, namespace, string
	},
	keywordstyle=\color{blue}\bfseries,
	otherkeywords={% Operators
		=>
	},
	identifierstyle=\color{black}\bfseries,
	sensitive=false,
	comment=[l]{//},
	morecomment=[s]{/*}{*/},
	commentstyle=\color{ForestGreen},
	stringstyle=\color{red},
	morestring=[b]',
	morestring=[b]",
	morestring=[b]`
}

%define colors for json
\colorlet{punct}{red!60!black}
\definecolor{delim}{RGB}{20,105,176}
\colorlet{numb}{magenta!60!black}

\lstdefinelanguage{json}{
    literate=
     *{0}{{{\color{numb}0}}}{1}
      {1}{{{\color{numb}1}}}{1}
      {2}{{{\color{numb}2}}}{1}
      {3}{{{\color{numb}3}}}{1}
      {4}{{{\color{numb}4}}}{1}
      {5}{{{\color{numb}5}}}{1}
      {6}{{{\color{numb}6}}}{1}
      {7}{{{\color{numb}7}}}{1}
      {8}{{{\color{numb}8}}}{1}
      {9}{{{\color{numb}9}}}{1}
      {:}{{{\color{punct}{:}}}}{1}
      {,}{{{\color{punct}{,}}}}{1}
      {\{}{{{\color{delim}{\{}}}}{1}
      {\}}{{{\color{delim}{\}}}}}{1}
      {[}{{{\color{delim}{[}}}}{1}
      {]}{{{\color{delim}{]}}}}{1},
}

\definecolor{background}{HTML}{EEEEEE}

\lstset{
	basicstyle=\ttfamily\footnotesize,
	frame=lines,
	captionpos=b,
	%rulecolor=\color{blue!80!black},
	inputencoding=utf8,
	tabsize=2,
	numbersep=5pt,
    showstringspaces=false,
	extendedchars=true,
	numbers=left,
	breaklines=true,
	backgroundcolor=\color{background},
}

\lstdefinestyle{sharpc}{
	language=CSharp,
	escapechar=`
}


\newenvironment{codeblock}{%
	\bigskip
	\noindent\begin{minipage}{\linewidth}%
}
		% Code
{%
	\end{minipage}
}


\newenvironment{codefloat}{%
	\begin{figure}
}
		% Code
{%
	\end{figure}
}

