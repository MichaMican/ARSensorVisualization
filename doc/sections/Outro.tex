\section{Schlussfolgerung und Ausblick}

Moderne Sicherheitskonzepte von Browsern sind nicht dazu ausgelegt,
in lokalen Netzwerken, Webseiten barrierefrei zu ermöglichen.
Daher ist es nicht möglich die Vorgaben, ohne Einschränkungen umzusetzen.
Diese Probleme können teilweise durch lokale Zertifikate gelöst werden,
was durch eine zukünftige Arbeit erzielt werden kann.

Das Testen von OpenGL Anwendungen im Browser erscheint, als ob eine
Ausführung auf einem Smartphone starke Performanzeinschnitte erleiden
würde. Dieser Eindruck lässt sich bei der dargestellten Anwendung nicht
bestätigen. Zukünftige Forschung in dieses Thema könnte Aufschlüsse geben.

Weitere Arbeiten können Backends produzieren, welche Visualisierungen
für andere Einsatzgebiete ermöglichen. Es kann zudem auch eine echte
Datenquelle, anstatt Simulationsdaten eingebunden werden.

Das Resultat der Arbeit zeigt einen guten Ansatz für die Arbeitsteilung
und Kommunikation zwischen Frontend und Backend. Die Visualisierung selbst
ist jedoch, auf einer 2-Dimensionalen Anzeigefläche, oft unklar.
Hier ist eine Ausarbeitung in stereoskopischer AR ein potentieller
Ansatz, dies zu lösen.

%\begin{itemize}
%	\item Vorgaben nicht umsetzbar ohne abstriche.
%	\item AR.js läuft besser auf handy als auf PC
%	\item Visualisierung is eig kake - Microsof MX wär doch viel geiler
%	\item Direkter Informationstransport zwischen frontend und backend muss umgesetzt werden (aktuell ja nur simulations daten)
%	\item HTTPS hosting in lokaler umgebung aufesetzbar
%	\begin{itemize} \renewcommand{\labelitemii}{$\Rightarrow$}
%		\item Gerät muss zertifikat vertrauen
%	\end{itemize}
%	
%	\item Andere implementierung des Backends
%	\begin{itemize}
%		\item Kann für Visualisierung in anderen Feldern helfen
%		\begin{itemize}
%			\item z. B. Luftströme bei Autos, etc.
%		\end{itemize}
%	\end{itemize}
%\end{itemize}
