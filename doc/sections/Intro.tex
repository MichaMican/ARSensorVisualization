\section{Einleitung}
Informationen aus der Umgebung aufzunehmen und zu verarbeiten sind seit je her ein Schlüsselaspekt zur Gewährleistung des Überlebens einer Spezies. Ob die Unterscheidung zwischen Essbarem und nicht Essbarem oder der Erkennung von Feinden oder sonstigen Gefahren, ein Lebewesen muss in kürzester Zeit ein Objekt analysieren können um über sein weiteres Vorgehen zu entscheiden. \newline
Hierbei gilt der Sehsinn als einer der wichtigsten Sinne in unserer bewussten Wahrnehmung (vgl. Majid 2018 \cite{Majid11369}). So ist der Mensch aus seiner Natur heraus dazu veranlagt über den visuellen Kanal sich am effektivsten Informationen zu erschließen. Diese Eigenheit wird in Bereichen wie der Werbeindustrie, der Unterhaltungsindustrie und in vielen weiteren genutzt. Oft ist hier jedoch ein statisches Medium das Problem. Ein Werbebanner kann sich nicht verändern, wenn man von einem anderen Winkel darauf blickt. Mit Virtual und Augmented Reality wurde mit den Jahren eine Technik geschaffen die diese Barriere durchbricht. Außerdem erreichen Virutal Reality (VR) und Augmented Reality (AR) mit der Zeit ein immer breiteres Publikum. Videospiele wie Pokemon GO nutzen das Smartphone als Projektionsfläche in die echte Welt. Der große Vorteil liegt hier in der Zugänglichkeit. Viele Nutzer haben bereits ein Handy und müssen sich keine weiteren Apparaturen kaufen um AR zu erleben.\newline
Diese Arbeit befasst sich mit der Umsetzung einer zugänglichen AR Lösung für das Smartphone um Verwirbelungen in flüssigem Metall einfach und verständlich anzuzeigen und um mit dieser Information
%evtl stat Information "Wissen" schreiben
die Qualität der Verarbeitung des Metalls zu verbessern. Das Ziel ist es die komplexen und abstrakten Messdaten greifbar im dreidimensionalen Raum auf eine Kokille zu Projizieren um so einen Blick in das Innere zu ermöglichen.